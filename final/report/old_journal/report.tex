\documentclass[a4paper, 12pt, titlepage]{article}
\usepackage{hyperref, listings,amssymb, amsmath, amsthm, graphicx, float}
\author{Joris Stork: 6185320 \and Lucas Swartsenburg: 6174388 \and Sander van
Veen: 6167969 \and Harm Dermois: 0527963}
\title{Laboratory log book: Robotics final project}
\begin{document}
\maketitle

\section{Objective} % {{{

Our objective was to combine the relatively advanced and portable communications 
and sensor functionality of wireless sensor network motes with the mobility 
and sensor functionality of three wheeled robots.

% }}}

\section{Method} % {{{


\subsection{Day 1: COM vintage} % {{{

We wanted to avoid re-inventing the wheel. Our department provided a
pre-existing driver for the Hemisson device, also provided, which sports a
microcontroller similar to that found on the jobot. The two devices have similar
functionality. Our intention was to alter the drivers for the Hemisson to work
with the jobot.

Our first challenge was to establish a connection between our PCs and the
Hemisson via the device's serial port. Since none of our PCs had a ``COM'' port,
we procured a USB to COM port adapter. During the first day our attempts to
establish a connection with the Hemisson device through this adapter failed. Our
attempts involved both a Linux PC and a Windows PC.

% }}}

\subsection{Day 2: Getting to grips} % {{{

Early during our second day we established that the COM port for the USB to
serial adapter needed to be set in the Windows device driver list. An analogous
solution applied in the case of the linux computer. With this obstacle out of
the way, we split our
group into two: Lucas and Harm were assigned the task of configuring the
SunSpots, and Sander and Joris were to port the C code that drives the Hemisson
to the jobots. 

\subsubsection{Joris and Sander} % {{{

Once we had established that the normal bootloader and demo code were on board
the Hemisson, we set about trying to drive the device over a serial connection.
To do this we used a Linux machine and the USB to serial port adapter. The
Hemisson's manual contained details of parameters for establishing a serial port 
connection to the Hemisson's MCU: baud rate of 115,200 bps; byte size of 8 bits; 
no parit
data; one stop bit; no flow control. We were successfully able to open a serial
port connection to the Hemisson MCU using the pySerial python module under
Linux. Using the command syntax outlined in the Hemisson manual, with commands
such as \texttt{D,-5,5} to drive the MCU's peripherals, our first successful
command over the serial port was to set the Hemisson's wheels in motion, with
the command: 
\begin{verbatim}D\r\end{verbatim}
We decided to pursue this route towards controlling the Hemisson from the linux
machine a little further in order to have a small framework from which to test
the Hemisson's and later the Jobot's configuration over the serial port.
See appendix \ref{python} for the python code we wrote to
control the Hemisson over the serial port.

% }}}

% }}}

\subsection{Day 3: Jobot in C} % {{{


\subsubsection{Joris and Sander} % {{{

We finished writing a python script to control the Hemisson over a serial port 
connection. Appendix \ref{python} is a verbatim printout of the
python code. The script contains a number of definitions, effectively functions
that translate higher level commands in a test code section of the script into
the necessary serial port data. The test code section as displayed in the
appendix instructs the Hemisson to drive forwards at half thrust, then to turn 

% }}}

% }}}

% }}}

\section{Hardware}
\begin{itemize}
    \item SunSpots: Wireless Sensor Network motes. See also 
    \href{http://en.wikipedia.org/wiki/Sun_SPOT}
        {this wikipedia article} - which includes some useful links. The sunspot
        includes the following:
        \subitem Microcontroller: $180 MHz$ 32 bit ARM920T core - $512K$ RAM -
        $4M$
        Flash;
        \subitem $2.4 GHz$ IEEE $802.15.4$ radio with integrated antenna;
        \subitem $AT91$ timer chip;
        \subitem USB interface;
        \subitem $2G/6G$ three-axis accelerometer;
        \subitem Temperature sensor;
        \subitem Light sensor;
        \subitem 8 tri-color LEDs;
        \subitem 6 analog inputs;
        \subitem 2 momentary switches;
        \subitem 5 general purpose I/O pins;
        \subitem 4 high current output pins;
        \subitem $3.7V$ rechargeable $750 mAh$ lithium-ion battery;
    \item jobot: Predesigned robot with three wheels in a triangular
    configuration, and including:
        \subitem a PIC16F452 microcontroller with max $40MHz$ CPU clock, 256 byte
        EEPROM data, $32KB$ program memory, and digital communication peripherals
        (1-A/E/USART, 1-MSSP(SPI/I2C));
    \item Hemisson: Predesigned robot with two wheels for mobility, that includes:
        \subitem a PIC16F877 microcontroller with $20MHz$ CPU clock, 8bit,
        $8K$ $\times$ 14 bit words program memory, 368 bytes data memory, 256 bytes
        EEPROM data memory, 14 interrupts, I/O ports A,B,C,D,E, three timers,
        serial communications (MSSP, USART), parallel communications (PSP), 8
        input channel, 10 bit analog to digital module;
        \subitem two DC motors for independent control of two wheel. Open loop
        control without encoders;
        \subitem eight IR ambient light sensors;
        \subitem six IR obstacle detection sensors;
        \subitem two line detection sensors;
        \subitem a standard $9V$ (PP3) battery connector;
        \subitem serial port with DB9 connector;
        \subitem a TV remote receiver;
        \subitem a buzzer;
        \subitem four LEDs;
        \subitem four programmable switches;
        \subitem an extension bus for extra modules;
    \item Linux, Mac OS X, Windows XP and Windows 7 driven PCs;
    \item USB to RS232 / DB9 converter. Make: K\"onig;
        

\end{itemize}

\section{Software}
% }}}

% }}}

\section{results} % {{{
% }}}

\section{assumptions} % {{{

% }}}

\newpage
\appendix

\section{Python code to control the Hemisson} % {{{
\label{python}
\begin{verbatim}
from serial import Serial, PARITY_NONE, EIGHTBITS, STOPBITS_ONE
from time import sleep

class HemissonException(Exception):
    def __init__(self, value):
        self.value = value

    def __str__(self):
        return repr(self.value)

class Hemisson:
    def __init__(self):
        """
        Initialise the serial connection to the Hemisson robot.
        """

        self.serial = Serial(0)
        self.serial.setParity(PARITY_NONE)
        self.serial.setByteSize(EIGHTBITS)
        self.serial.setStopbits(STOPBITS_ONE)
        self.serial.setBaudrate(115200)

        print 'i Initialised serial connection "%s".' % self.serial.portstr
        self.serial.write(chr(254)+'\r')
        self.readline()
        self.readline()
        self.readline()

    def __delete__(self):
        """
        Disconnect the serial connnection to the Hemisson robot.
        """

        print 'i Destroying serial connection "%s".' % self.serial.portstr
        self.write(chr(8))
        self.serial.close()

    def beep(self, state):
        """
        Generates a continuous beep, depending on state (0 = Off, 1 = On).
        """

        if state not in (0, 1):
            raise HemissionException('Beep state should be 0 (Off) or 1 (On).')

        self.write('H,%d' % state)
        self.readline()

    def drive(self):
        """
        Drive forward (setting both wheel's drive speed to '2').
        """

        self.set_speed(2)

    def get_switches(self):
        """
        Read the current status of the four top switches. Possible values are 0
        (= robot's right handside) and 1 (= robot's left handside). The first
        value is the value of the first switch from the front of the robot.
        """

        self.write('I')
        self.readline()

    def set_speed(self, left, right=None):
        """
        Set driving speed of left and right wheel. If only the left wheel drive
        speed is given, the right wheel's drive speed is set to the left
        wheel's drive speed.
        """

        if right == None:
            right = left

        if not( -9 <= left <= 9 and -9 <= right <= 9):
            raise HemissonException(
                'Keep the wheel drive speed value between -9 and 9.')

        self.write('D,%d,%d' % (left, right))
        self.readline()

    def stop(self):
        """
        Stop moving forward (setting both wheel's drive speed to zero).
        """

        self.write('D,0,0')
        self.readline()

    def readline(self):
        """
        Read a single newline terminated line from the serial connection.
        """

        print '< %s' % self.serial.readline()[:-1]

    def remote_version(self):
        """
        Display version of the HemiOS running on the connected Hemisson robot.
        """

        self.write('B')
        self.readline()

    def reset(self):
        """
        Reset the robot's processor as if the On/Off switch is cycled.
        """

        self.serial.write('Z')
        self.readline()

    def write(self, msg):
        """
        Write a message through the serial connection to the Hemisson robot.
        """

        print '> %s\\n\\r' % msg
        self.serial.write(msg+'\n\r')

if __name__ == '__main__':
    robot = Hemisson()
    robot.remote_version()
    robot.get_switches()

    #robot.drive()
    for i in range(4):
        robot.set_speed(4)
        sleep(2)
        robot.set_speed(-4,4)
        sleep(1.5)

    robot.stop()
    #robot.beep(1)
    #sleep(2)
    #robot.beep(0)
    #del robot

\end{verbatim}

% }}}

\section{Appendix B: microcontroller program source} % {{{

\begin{verbatim}

\end{verbatim}

% }}}

\section{Appendix C: sunspot source} % {{{

% }}}

\end{document}
