\documentclass[a4paper, 12pt, titlepage]{article}
\usepackage{listings,amssymb, amsmath, amsthm, graphicx, float, hyperref}
\hypersetup{
    colorlinks=false,
    pdfborder={0 0 0},
}
\newcommand{\tsup}[1]{\ensuremath{\sp{\text{#1}}}}

\author{Harm Dermois: 0527963\and Joris Stork: 6185320\and Lucas Swartsenburg: 6174388\and Sander van Veen: 6167969}

\title{Cobots (and Sun SPOTS)}

\begin{document}
\maketitle
\tableofcontents
\newpage

\begin{abstract}
\end{abstract}

\section{Introduction}

This project concludes our team's involvement in a January 2011 
robotics course at the University of Amsterdam (UvA). The team consists 
of four second-year UvA students enrolled for the BSc in Computer Science. The
overall aim of this project is to address the challenges inherent in building a
microcontroller based mobile robot.

Our team was given the task, firstly, of ``ressuscitating'' the UvA's
Java-programmable ``Jobot'' robots, whose software development platform 
had become defunct and inoperative. Our second task was to combine the 
Jobots' mobility and sensing capabilities with the enhanced communications, 
sensing and processing functionality of 
Sun SPOTS -- wireless sensor network devices from the Sun Labs
stable. In this way we hoped to achieve a few things, namely to:

\begin{itemize}
    \item write, compile and load programs from scratch for a PIC based
    microcontroller board;
    \item create drivers for the servos and sensors connected the the above
    board;
    \item further explore the Sun SPOTS' sensing and communications
    characteristics through those devices' Java frameworks;
    \item build a data transport between the Sun
    SPOTS and the standard microcontrollers on the Jobots, possibly over 
    an $I^2C$ bus;
    \item design an interface between our respective software implementations
    on the Jobot and on the Sun SPOTS;
    \item build a remote control system using the above interface;
    \item attempt some basic robot activities using our adapated, Sun SPOT
    controlled robot(s), such as: obstacle avoidance; reaching a goal;
    recognising a fellow robot; following a fellow robot, and so on, and;
    \item learn!
\end{itemize}

Strict time and resource limitations took precedence over any adherence
to pre-formulated goals and methods. That said, our project was successful,
and our team achieved a bonus on the way.

This report describes the principal materials at our disposal and our approach
to completing the given project tasks. Our results are described per
challenge, in chronological order of their occurence. 
This approach reflects the exploratory
nature of the project, in which a string of unexpected obstacles dictated 
the structure of our efforts over the past two weeks. 

\section{Materials and approach}
Appendix 

\section{Results}

\subsection{Cobot}

\subsection{SunSpots}

\section{Discussion}

\section{Conclusion}

\Appendix

\bibliographystyle{plain}
\bibliography{cited}
\end{document}
