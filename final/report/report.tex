\documentclass[a4paper, 12pt, titlepage]{article}
\usepackage{listings,amssymb, amsmath, amsthm, graphicx, float, hyperref}
\hypersetup{
    colorlinks=false,
    pdfborder={0 0 0},
}
\newcommand{\tsup}[1]{\ensuremath{\sp{\text{#1}}}}

\author{Harm Dermois: 0527963\and Joris Stork: 6185320\and Lucas Swartsenburg: 6174388\and Sander van Veen: 6167969}

\title{Cobots (and Sun SPOTS)}

\begin{document}
\maketitle
\tableofcontents
\newpage

\begin{abstract}
\end{abstract}

\section{Introduction}

This project concludes our team's involvement in a January 2011 
robotics course at the University of Amsterdam (UvA). The team consists 
of four second-year UvA students enrolled for the BSc in Computer Science. The
overall aim of this project is to address the challenges inherent in building a
microcontroller based mobile robot.

Our team was given the task, firstly, of ``ressuscitating'' the UvA's
Java-programmable ``Jobot'' robots, whose software development platform 
had become defunct and inoperative. Our second task was to combine the 
Jobots' mobility and sensing capabilities with the enhanced communications, 
sensing and processing functionality of 
Sun SPOTS -- wireless sensor network devices from the Sun Labs
stable. In this way we hoped to achieve a few things, namely to:

\begin{itemize}
    \item write, compile and load programs from scratch for a PIC based
    microcontroller board;
    \item create drivers for the servos and sensors connected the the above
    board;
    \item further explore the Sun SPOTS' sensing and communications
    characteristics through those devices' Java frameworks;
    \item build a data transport between the Sun
    SPOTS and the standard microcontrollers on the Jobots, possibly over 
    an $I^2C$ bus;
    \item design an interface between our respective software implementations
    on the Jobot and on the Sun SPOTS;
    \item build a remote control system using the above interface;
    \item attempt some basic robot activities using our adapated, Sun SPOT
    controlled robot(s), such as: obstacle avoidance; reaching a goal;
    recognising a fellow robot; following a fellow robot, and so on, and;
    \item learn!
\end{itemize}

Strict time and resource limitations took precedence over any adherence
to pre-formulated goals and methods. That said, our project was successful,
and our team achieved a bonus on the way.

This report describes the principal materials at our disposal and our approach
to completing the given project tasks. Our results are described per
challenge, in chronological order of their occurence. 
This approach reflects the exploratory
nature of the project, in which a string of unexpected obstacles dictated 
the structure of our efforts over the past two weeks. 

\section{Methods}

Appendix \ref{hardware} provides a more exhaustive description of the various
pieces of hardware used in this project. This section provides an overview:

\section{Hardware}
Our 

\section{Results}
\subsection{Day 1: COM vintage} % {{{

We wanted to avoid re-inventing the wheel. Our department provided a
pre-existing driver for the Hemisson device, also provided, which sports a
microcontroller similar to that found on the jobot. The two devices have similar
functionality. Our intention was to alter the drivers for the Hemisson to work
with the jobot.

Our first challenge was to establish a connection between our PCs and the
Hemisson via the device's serial port. Since none of our PCs had a ``COM'' port,
we procured a USB to COM port adapter. During the first day our attempts to
establish a connection with the Hemisson device through this adapter failed. Our
attempts involved both a Linux PC and a Windows PC.

% }}}

\subsection{Day 2: Getting to grips} % {{{

Early during our second day we established that the COM port for the USB to
serial adapter needed to be set in the Windows device driver list. An analogous
solution applied in the case of the linux computer. With this obstacle out of
the way, we split our
group into two: Lucas and Harm were assigned the task of configuring the
SunSpots, and Sander and Joris were to port the C code that drives the Hemisson
to the jobots. 


Once we had established that the normal bootloader and demo code were on board
the Hemisson, we set about trying to drive the device over a serial connection.
To do this we used a Linux machine and the USB to serial port adapter. The
Hemisson's manual contained details of parameters for establishing a serial port 
connection to the Hemisson's MCU: baud rate of 115,200 bps; byte size of 8 bits; 
no parit
data; one stop bit; no flow control. We were successfully able to open a serial
port connection to the Hemisson MCU using the pySerial python module under
Linux. Using the command syntax outlined in the Hemisson manual, with commands
such as \texttt{D,-5,5} to drive the MCU's peripherals, our first successful
command over the serial port was to set the Hemisson's wheels in motion, with
the command: 
\begin{verbatim}D\r\end{verbatim}
We decided to pursue this route towards controlling the Hemisson from the linux
machine a little further in order to have a small framework from which to test
the Hemisson's and later the Jobot's configuration over the serial port.
See appendix \ref{python} for the python code we wrote to
control the Hemisson over the serial port.

% }}}

% }}}

\subsection{Day 3: Jobot in C} % {{{


We finished writing a python script to control the Hemisson over a serial port 
connection. Appendix \ref{python} is a verbatim printout of the
python code. The script contains a number of definitions, effectively functions
that translate higher level commands in a test code section of the script into
the necessary serial port data. The test code section as displayed in the
appendix instructs the Hemisson to drive forwards at half thrust, then to turn 


\section{Discussion}

\section{Conclusion}

\newpage
\appendix

\section{Hardware in detail}
\label{hardware}
\begin{itemize}
    \item SunSpots: Wireless Sensor Network motes. See also 
    \href{http://en.wikipedia.org/wiki/Sun_SPOT}
        {this wikipedia article} - which includes some useful links. The sunspot
        includes the following:
        \subitem Microcontroller: $180 MHz$ 32 bit ARM920T core - $512K$ RAM -
        $4M$
        Flash;
        \subitem $2.4 GHz$ IEEE $802.15.4$ radio with integrated antenna;
        \subitem $AT91$ timer chip;
        \subitem USB interface;
        \subitem $2G/6G$ three-axis accelerometer;
        \subitem Temperature sensor;
        \subitem Light sensor;
        \subitem 8 tri-color LEDs;
        \subitem 6 analog inputs;
        \subitem 2 momentary switches;
        \subitem 5 general purpose I/O pins;
        \subitem 4 high current output pins;
        \subitem $3.7V$ rechargeable $750 mAh$ lithium-ion battery;
    \item jobot: Predesigned robot with three wheels in a triangular
    configuration, and including:
        \subitem a PIC16F452 microcontroller with max $40MHz$ CPU clock, 256 byte
        EEPROM data, $32KB$ program memory, and digital communication peripherals
        (1-A/E/USART, 1-MSSP(SPI/I2C));
    \item Hemisson: Predesigned robot with two wheels for mobility, that includes:
        \subitem a PIC16F877 microcontroller with $20MHz$ CPU clock, 8bit,
        $8K$ $\times$ 14 bit words program memory, 368 bytes data memory, 256 bytes
        EEPROM data memory, 14 interrupts, I/O ports A,B,C,D,E, three timers,
        serial communications (MSSP, USART), parallel communications (PSP), 8
        input channel, 10 bit analog to digital module;
        \subitem two DC motors for independent control of two wheel. Open loop
        control without encoders;
        \subitem eight IR ambient light sensors;
        \subitem six IR obstacle detection sensors;
        \subitem two line detection sensors;
        \subitem a standard $9V$ (PP3) battery connector;
        \subitem serial port with DB9 connector;
        \subitem a TV remote receiver;
        \subitem a buzzer;
        \subitem four LEDs;
        \subitem four programmable switches;
        \subitem an extension bus for extra modules;
    \item Linux, Mac OS X, Windows XP and Windows 7 driven PCs;
    \item USB to RS232 / DB9 converter. Make: K\"onig;
        

\end{itemize}

\bibliographystyle{plain}
\bibliography{cited}
\end{document}
