\documentclass{beamer}
\usepackage{hyperref}

\begin{document}
\section{Topological Maps}
\begin{frame}
 \textbf{Topological maps}: describe the environment as a graph that connectecs specific locations in the world and represents them as vertices.
 \begin{enumerate}
 	\item Because metric representations cost too much memory to maintain in the long run.
 	\item Easy to understand for humans.
 	\item The nodes on the graphs are landmarks or features of the environment. The edges are paths between the different nodes. 
 	\item Landmarks can be artificial or natural(doorways, signs, marker.)
 	\item Can be extended by direction, terrain type, behaviors needed to navigate the path.
 	\item Landmarks can look the same so you need to make sure you dont use two or more nodes to represent the same landmark.
 \end{enumerate}
\end{frame}

\begin{frame}{Marker based Exploration}
 \begin{enumerate}
 	\item There isn't always a signature to distinguish two landmarks from eachother.
 	\item The robot needs to have something to mark where it has already been.(spray paint, bread crumbs)
 	\item Use unique markes which it it can pick up drop and recognize.
 \end{enumerate}
\end{frame} 

\begin{frame}{Marker based exploration algorithm}
	\begin{enumerate}
		\item
		\item
	\end{enumerate}
\end{frame}

\section{Multiple Robots}

\begin{frame}{Multiple Robots}
	\textbf{Why would you use multiple Robots}
	\begin{enumerate}
		\item \textbf{Improved Robustness}: A multirobot can ,in principle, keep functioning even if one indiviudual robots dail completely. 
		\item \textbf{Improved effiency}: It is possible for a group of robots to accomplish a search or exploration task faster than an equivalent single robot.
		\item \textbf{Alternative Algorithms}: For some tasks, the availability of multiple robots allows feadible or guaranteed algorithms to be implemented when no such algorithm is 		available for a single robot system. 
	\end{enumerate} 
\end{frame}

\begin{frame}{Multi Robots in Practice}
	\textbf{Problems}
	\begin{enumerate}
		\item \textbf{Where are teh other robots?}: Rendezvous with other robots
		\item \textbf{Partitioning}: Finding a good way to distribute the work amongst the robots.
		\item \textbf{Multi-robot planning}: Prevent the trajectories of the robots to collide. 
		\item \textbf{Merging the data from the indivual team}:
	\end{enumerate}	
\end{frame}

\begin{frame}{Rendezvous}
	\textbf{Rendezvous}: Is a having two or more robots meet at an appointed place and time. 
	\begin{enumerate}
		\item Rendezvous is needed for robots that can only communicate in close proximities, but may also be needed to exchange objects between robots.
		\item When Multiple robots try to complete a task collaborotavely without prior knowledge. They need to to exchange information while they are still working at the task at hand.  
		\item	 If they dont meet they cannot benefit from what others have already learned. 
	\end{enumerate}
\end{frame}

\begin{frame}{Too many Rendezvous}
	\textbf{Problems}:Robots mustn't devote too much energy to rendezvous
		\begin{enumerate}			
			\item The extent to which the two robots agree on their perceptions of the environment. What is the difference 
			\item The degree of synchornization of the robots can attain expressed as the likelihood that an appointed rendezvous at a common location will fail owning to a failure to arrive at the same time
			\item The extent Of the commonality between the region of space the robots have explored.
		\end{enumerate}
	\textbf{There are many different rendezvous algorithms}
	\begin{enumerate}
		\item \textbf{plan based}: 
		\item \textbf{stochastic algorithmes}: One stationary and one seeks or Randomly visit rendezvous points which are points or interest.
	\end{enumerate}
\end{frame}

\begin{frame}{Map fusion}
	\textbf{Exploration}: is needed when the problem doesn't involve foraging.
	\begin{enumerate}
		\item Is needed to make the collaborative effort worthwhile.
		\item Complexity of the map-merging depends on the , Odemetry error, the fidelity of the sensing used. and the richness of the evironment.
		\item Fusing maps using cross correlation depends on teh fact that the individual maps overlap "sufficiently'.
		\item Done by rotation and translating of the given maps.
	\end{enumerate}	
\end{frame}

\begin{frame}{Exploration with Multiple robots}
	\begin{enumerate}
		\item Robots are only allowed to commnicate when they are in the same node.
		\item Split all the work between all the robots and have them explore their own part of the graph.
		\�tem Plan rendezvous, to harmonize the information they got till then and make a single consistent representation of the environment they are in.
		\item Redevide the work and repeat this till everything is known. 
	\end{enumerate}
\end{frame}
\begin{frame}

	\url{http://www.csupomona.edu/~ftang/courses/CS499/notes/navigation3.pdf }
\end{frame}

\end{document}