\documentclass{beamer}
\usetheme{Goettingen}
\usepackage{hyperref, graphicx}%, cite}
\begin{document}
\title{Chapter 8: Maps and related tasks}  
\author{Harm Dermois \and Joris Stork}
\date{\today} 

\frame{\titlepage} 
\frame{\tableofcontents} 

\section{Robots and maps} 

\frame{\frametitle{Maps for robots}
\begin{columns}[c]
\column{5cm}
Human maps:
\begin{itemize}
    \item Often unavailable; 
    \item Often incomplete: 
    \begin{itemize}
        \item human relevant data; 
        \item robot relevant data;
    \end{itemize} 
\end{itemize}
\column{5cm}
\includegraphics[width=5cm]{blueprint.png}
\end{columns}
}

\frame{\frametitle{Maps for robots}
\begin{columns}[c]
\column{5cm}
Human maps:
\begin{itemize}
    \item Often unavailable; 
    \item Often incomplete: 
    \begin{itemize}
        \item human relevant data;
        \item robot relevant data;
    \end{itemize} 
    \item Wrong level(s) of abstraction: human oriented; 
\end{itemize}
\column{5cm}
\includegraphics[width=5cm]{google_pizzas.png}
\end{columns}
}

\frame{\frametitle{Maps by robots}
\begin{itemize}
    \item
        Building a robot friendly map is very difficult and tedious.
    \item
        Robots are good candidates to build maps with and for their own sensory
        suite;
\end{itemize}
Conclusion: Design robots to autonomously construct, update and validate
maps destined for robot use.
}

\frame{\frametitle{Map paradigms: metric vs. topological}
\begin{columns}[c]
\column{5cm}
\begin{itemize}
    \item Metric 
        \begin{itemize}
        \item Sensorial
        \item Geometric
        \end{itemize}            
\end{itemize}
\column{5cm}
\includegraphics[width=5cm]{metric.png}
\end{columns}
}

\frame{\frametitle{Map paradigms: metric vs. topological}
\begin{columns}[c]
\column{5cm}
\begin{itemize}
    \item Metric 
        \begin{itemize}
        \item Sensorial
        \item Geometric
        \end{itemize}            
    \item Topological 
        \begin{itemize}
        \item Local relational
        \item Topological
        \item Semantic
        \end{itemize}            
\end{itemize}
\column{5cm}
\includegraphics[width=5cm]{topological.png}
\end{columns}
}

\frame{\frametitle{Direction of map hierarchy}
\begin{itemize}
    \item Giralt et al.%\cite{giralt1979multi}
    : metric to topological. 
    \item Kuipers and Levitt %\cite{kuipers1991robot}
    : topological to metric. Low level topological
    landmarks as starting point.
\end{itemize}
}

\frame{\frametitle{Types of data}
\begin{itemize}
    \item (Derived) spacial occupancy; 
    \item (Direct) sensor measurements in relation to position. e.g. olfaction.
\end{itemize}
}

\section{Sensorial maps}

\frame{\frametitle{Sensorial maps}
\begin{itemize}
    \item Represent sensor measurements against odometry; 
    \item Collection of measurements: $[I_i(x_i,y_i,\theta_i)]$
\end{itemize}
}

\subsection{Image based mapping}
\frame{\frametitle{Image based mapping}
The challenge: 
\begin{itemize}
    \item how to sample the set of possible measurements,
    $\lbrace I_i\rbrace$;
    \item how to turn the samples into a continuous $I$.
\end{itemize}
}

\frame{\frametitle{Li: street panoramas}
Li et al.: robots builds graph representing street network:
\begin{itemize}
    \item edges = streets;
    \item nodes = intersections;
\end{itemize}
Robot collects by:
\begin{itemize}
    \item moving in a closed loop, always turning left;
    \item recording panoramas of left and right sides of streets;
    \item concluding a loop by identifying previously recorded street side;
\end{itemize}
}

\frame{\frametitle{Bourque: robot sightseeing}
Bourque et al.%\cite{bourque2000automated}
: robots builds graph nodes corresponding to panorama shots, in a
less constrained environment, by:
\begin{itemize}
    \item choosing sample (panorama) points based on models of human attention;
    \item using ``alpha backtracking'' to make trade-off between distance to
    next sample point and optimality of next sample point;
\end{itemize}
}

\subsection{Spacial occupancy representations}

\frame{\frametitle{Spacial occupancy grid}
Pioneered by Elfes and Moravec %\cite{}
\begin{itemize}
    \item Grid of pixels
\end{itemize}
}

\frame{\frametitle{Spacial occupancy grid}
\begin{columns}
\column{5cm}
\begin{itemize}
    \item Grid of pixels
    \item Volume of voxels
\end{itemize}
\column{5cm}
\includegraphics[width=5cm]{voxels.png}
\end{columns}
}

\frame{\frametitle{Spacial occupancy: data represented}
\begin{itemize}
    \item Fill pixels / voxels with degree of occupancy data.
    \item More refined: fill pixels / voxels with probability of occupancy data.
\end{itemize}
}

\frame{\frametitle{Spacial occupancy: probabilistic approach}
\begin{itemize}
    \item Example of a laser sensor: probability of an actual distance $z$ for a given laser
    reading $r$ computed using Bayes' theorem:
    \begin{eqnarray}
        P(z\lvert r) = \frac{P(z) P(r \lvert z)}{P(r)}
        \nonumber
    \end{eqnarray}
    
    \pause
    \item Generalised:
    \begin{eqnarray}
        P(W\lvert R) = \frac{P(W) P(R \lvert W)}{P(R)}
        \nonumber
    \end{eqnarray}
    with $P(R_i)=\sum_j P(R_i \lvert W_j) P (W_j)$
\end{itemize}
}

\frame{\frametitle{Spacial occupancy: probabilistic approach}
The result: \em{maximum a posteriori }\em (MAP). World model that most reasonably
estimates environment according to Bayesian approach.\\
\pause
Considerations:
\begin{itemize}
    \item very general: no assumed model, deals with multiple sensors;
    \item requires accurate probabilistic model of the sensors;
    \item requires a lot of memory for the occupancy map;
    \item measurement locations/times discarded: geometric accuracy reduced;
    \item important to avoid accumulated positional errors - e.g. by iteratively
    recomputing position;
    \item needs an exploration policy: e.g. random or towards ``unkown' 'areas.
\end{itemize}
}

\frame{\frametitle{Spacial occupancy: Markov models}
    Markov localisation: estimating robot's location based on sensor data by
    maintaining probability density grid for the robot's environment, with each
    cell representing a possible robot pose.
}

\subsection{Geometric maps}

\frame{\frametitle{Geometric maps}
Accurate, with two assumptions:
\begin{itemize}
    \item sensor data is suitable
    \item environment is suitable
\end{itemize}
}

\frame{\frametitle{Geometric maps: exploration}
Challenge is exploration. Includes searching for:
\begin{itemize}
    \item a goal position;
    \item route with specific properties;
    \item ``covering'' a space;
    \item occupancy.
\end{itemize}
}

\frame{\frametitle{Geometric maps: reach goal}
Papadimitriou and Yannakakis's bug-like algorithm for reaching known goal 
from known origin in unkown environment with obstacles:
\begin{itemize}
    \item move ``towards'' line connecting origin and goal;
    \item if not possible, move in arbitrary direction;
\end{itemize}
Useful in certain simple types of environment, notably where obstacles are:
\begin{itemize}
    \item rectilinear; 
    \item nonintersecting;
    \item aligned with world coordinates.
\end{itemize}
In more general environments no bound is possible.
}

\frame{\frametitle{Geometric maps: geometric representations}
   Chosen geometric representation influences applicable algorithms. Important
   representation is that of ``street polygons''.
}

\frame{\frametitle{Geometric maps: Street polygons}
    \begin{columns}[c]
    \column{5cm}
    Polygon such that:
        \begin{itemize}
        \item there is a start vertex $S$ and end vertex $T$
        \item vertices and lines categorised as ``left'' or ``right'' with
        respect to line segment from $S$ to $T$;
        \item every vertex on either side is visible to some vertex on
        the other
    \end{itemize}
    \column{5cm}
        \includegraphics[width=5cm]{streetpolygon.png}
    \end{columns}
}

\frame{\frametitle{Geometric maps: Street polygons}
    
}

\section{Topological Maps}

\frame{\frametitle{Topological Maps}
 \textbf{Topological maps}: describe the environment as a graph that connectecs specific locations in the world and represents them as vertices.
 \begin{enumerate}
 	\item Because metric representations cost too much memory to maintain in the long run.
 	\item Easy to understand for humans.
 	\item The nodes on the graphs are landmarks or features of the environment. The edges are paths between the different nodes. 
 	\item Landmarks can be artificial or natural.
 	\item Landmarks can look the same so you need to make sure you dont use two or more nodes to represent the same landmark.
 \end{enumerate}
}

\frame{\frametitle{Marker based exploration}
 \begin{enumerate}
 	\item No prior information about the environment available.
 	\item Can be extended by enumerating the edges incident to the node entered. Edge you traveled along is 0 and enumerate clockwise. This enumeration is local cause it depends on the edge the robot moved over.
 	\item Landmarks aren't distinguishable from eachother.
 	\item The robot needs to have something to mark where it has already been.(spray paint, bread crumbs)
 	\item Use unique marks which it can pick up, drop and recognize.
 \end{enumerate}
} 

\frame{\frametitle{Marker based exploration algorithm}
	\begin{enumerate}
		\item Builds up the known graph by traveling along the incident edges.
		\item $v_{i}$ is the node where the robot is currently at. $v_{j}$ is the node where the robot is moving to. $E_{i,j}$ is the edge between the 2 nodes.
		\item Transition function need to follow these properties. If $(v_{i},E_{i,j},r) = v_{j} $ and $(v_{j},E_{i,j},s) = v_{k} $, then $v_{j},E_{i,j},-s) = v_{i}$
		\item Moves are invertible and can be retraced.
		\item $t \neq -s$ then $v_{j},E_{i,j},-s) = v_{i}$ and  $(v_{j},E_{i,k},s) = v_{j}$ are not valid. To avoid redundant and degenerate paths.
	\end{enumerate}
}

\frame{\frametitle{Marker based exploration algorithm(2)}
	\textbf{Operations for the robot}
	\begin{enumerate}
		\item r stands for move along the given edge.
		\item Each marker can be in 3 different states[pickup,putdown,null]
		\item Marker based perception: At each vertex the robot can see two things [present, not-present]
	\end{enumerate}
	\includegraphics[scale= 0.5]{edgeordering.png}
}

\frame{\frametitle{Marker based exploration algorithm (3)}
	\textbf{Operations for the robot}
	\begin{enumerate}
		\item Robot can determine the relative posotions of the edges by enumerating the edges like said before. 
		\item Entering the same vertex from a different edge gives 2 different ordering. The robot needs to make a global ordering. 
		\item Subgraph S for explored edges and U for unexplored are incident to unknow nodes.
		\item First validate all explored nodes. Make sure there aren't any doubles by looking for markers.
		\item	If there is no marker found at a certain node v add it the subgraph S and add the edge which was taken aswell.
		\item Enumerate all edges incident to the new node and add them to U. 
		\item Do this till subgraph U is empty.
	\end{enumerate}
}

\frame{{Example}
	\includegraphics[scale =0.3]{voorbeeld.png}
}

\section{Multiple Robots}
\frame{\frametitle{Multiple Robots}
	\textbf{Why would you use multiple Robots}
	\begin{enumerate}
		\item \textbf{Improved Robustness}: A multirobot can ,in principle, keep functioning even if one indiviudual robots dail completely. 
		\item \textbf{Improved effiency}: It is possible for a group of robots to accomplish a search or exploration task faster than an equivalent single robot.
		\item \textbf{Alternative Algorithms}: For some tasks, the availability of multiple robots allows feadible or guaranteed algorithms to be implemented when no such algorithm is 		available for a single robot system. 
	\end{enumerate} 
}

\frame{\frametitle{Multi Robots in Practice}
	\textbf{Problems}
	\begin{enumerate}
		\item \textbf{Where are the other robots?}: Rendezvous with other robots
		\item \textbf{Partitioning}: Finding a good way to distribute the work amongst the robots.
		\item \textbf{Multi-robot planning}: Prevent the trajectories of the robots to collide. 
		\item \textbf{Merging the data from the indivual team}:
	\end{enumerate}	
}

\frame{\frametitle{Rendezvous}
	\textbf{Rendezvous}: is having two or more robots meet at an appointed place and time. 
	\begin{enumerate}
		\item Rendezvous is needed for robots that can only communicate in close proximities, but may also be needed to exchange objects between robots.
		\item When Multiple robots try to complete a task collaborotavely without prior knowledge. They need to to exchange information while they are still working at the task at hand.  
		\item	 If they dont meet they cannot benefit from what others have already learned. 
	\end{enumerate}
}

\frame{\frametitle{Too many Rendezvous}
	\textbf{Problems}: Robots mustn't devote too much energy to rendezvous
		\begin{enumerate}			
			\item The extent to which the two robots agree on their perceptions of the environment. What is the difference 
			\item The degree of synchornization of the robots can attain expressed as the likelihood that an appointed rendezvous at a common location will fail owning to a failure to arrive at the same time
			\item The extent Of the commonality between the region of space the robots have explored.
		\end{enumerate}
	\textbf{There are many different rendezvous algorithms}
	\begin{enumerate}
		\item \textbf{plan based}: 
		\item \textbf{stochastic algorithms}: One stationary and one seeks or Randomly visit rendezvous points which are points or interest.
	\end{enumerate}
}

\frame{\frametitle{Map fusion}
	\textbf{Map fusion}: is needed when the problem doesn't involve foraging.
	\begin{enumerate}
		\item Is needed to make the collaborative effort worthwhile.
		\item Complexity of the map-merging depends on the , Odemetry error, the fidelity of the sensing used. and the richness of the evironment.
		\item Fusing maps using cross correlation depends on teh fact that the individual maps overlap "sufficiently'.
		\item Done by rotation and translating of the given maps.
	\end{enumerate}	
}

\frame{\frametitle{Exploration with Multiple robots}
	\begin{enumerate}
		\item Robots are only allowed to commnicate when they are in the same node.
		\item Split all the work between all the robots and have them explore their own part of the graph.
		\item Plan rendezvous, to harmonize the information they got till then and make a single consistent representation of the environment they are in.
		\item Redevide the work and repeat this till everything is known. 
	\end{enumerate}
}

\frame{\frametitle{links}

	\url{http://www.csupomona.edu/~ftang/courses/CS499/notes/navigation3.pdf }
}

%\bibliographystyle{plain}
%\bibliography{cited}
\end{document}
