
\documentclass[a4,english]{article}
\usepackage{babel}

\begin{document}
\section{The Sun Spot}
The Sunspot is a vital part of our project. We use the Sun Spot to control the robots and intially it was our plan to have communication between the robots via these Sun Spots. Also the information from the Robot's  sensors were to be sent to the Sun Spots to be interpreted and handle accordingly.
\subsection{Communication between the spots} 
The first order of business was getting the Sun Spots to talk to eachother. To do this you need to open a function on the board of the Sun Spot to start listening for information you might get from other Sun Spots. It is almost always required to filter some of these packages, because if there are other Sun Spots communicating with eachother, you need to be sure you are receiving the right packages. We havn't implemented this, because we were working in a secluded area away from other sunspots which might interfer with out Sun Spots, but this is easily implemented. \\

We send some random information in the form of tilting the Sun Spot. This chosen to figure out both how to send the information and to get to know how the sensors of the Sun Spot function. We wanted the communication between to be both ways. So we had to use threads to get this working properly. Wihtout threads you will get stuck often because of desynchronization. 

We finally managed to corectly send the information between the two Sun Spot and we build the fundations of the communications between the two robots or as application as a remote control. With this basis it could be used to implement any sort of data transfer in an orderly fashion.  

\subsection{Bit-banging}
Bit-banging can be implemented at very low cost, and is used in, for example, embedded systems.\\

First we thought we could send the data between the Sun Spot and the CoBot via the the $I^2C$ bus, but half way through the project. It was found out this was't possible, because of technical issues. Namely the Sun Spot can only handle a voltage of 3V and the CoBot can only send signals of 5V. This is a big problem if you want to communicate between the two without frying the Sun Spot. So we had to use a technique called big-banging. Bit-banging is a technique for serial communications using software instead of dedicated hardware. This means you got to send all the information bit by bit instead of sending a full package of bytes in the form on any given data type. If you want to use this method it is needed to have good control of how and when the information is sent. This means you are responsible for all the timings and synchronisations between the parties. The good thing about  bit-banging is that it is low cost. it gives you more control and the best thing is that it can be used in every languages and can be used on every system that can communicate with software, because it is so low level. 

In our case we started by first sending information between two Sun Spots. We wanted to implement big-banging without the use of a clock. So there's a handshaking protocol need to get the two alligned. Here goes a picture of our handshaking protocol so just a picture of when each line is high or low. 
As you can expect is the hardest part of implementing this are synchronisations. As you can see in the picture the basic idea is really simple, but youalways need to make sure it really sent. insert something about the sending the information twice to makes sure it is ok data.


\end{document}