\documentclass[a4paper, 12pt, titlepage]{article}
\usepackage{hyperref, listings,amssymb, amsmath, amsthm, graphicx, float}
\author{Joris Stork: 6185320 \and Lucas Swartsenburg: 6174388 \and Sander van
Veen: 6167969 \and Harm Dubois: 0527963}
\title{Laboratory log book: Robotics final project}
\begin{document}
\maketitle

\section{Objective} % {{{

Our objective was to combine the functionality of SunSpot wireless sensor
motes with that of the ``jobot'' devices provided by the University of
Amsterdam. In order to demonstrate some of the key capabilities achieved in this
way, we aimed to configure and program the devices to carry out a ``game''
involving multiple mobile autonomous agents.

% }}}

\section{Method} % {{{

\subsection{Writing the jobot drivers} % {{{

\subsubsection{Day 1: COM vintage} % {{{

We wanted to avoid re-inventing the wheel. Our department provided a
pre-existing driver for the Hemisson device, also provided, which sports a
microcontroller similar to that found on the jobot. The two devices have similar
functionality. Our intention was to alter the drivers for the Hemisson to work
with the jobot.

Our first challenge was to establish a connection between our PCs and the
Hemisson via the device's serial port. Since none of our PCs had a ``COM'' port,
we procured a USB to COM port adapter. During the first day our attempts to
establish a connection with the Hemisson device through this adapter failed. Our
attempts involved both a Linux PC and a Windows PC.

% }}}

\subsubsection{Day 2: Getting to grips} % {{{
Early during our second day we established that the COM port for the USB to
serial adapter needed to be set in the Windows device driver list. We split our
group into two: Lucas and Harm were assigned the task of configuring the
SunSpots, and Sander and Joris were to port the C code that drives the Hemisson
to the jobots. 

% }}}

% }}}

% }}}

\section{Hardware}
\begin{itemize}
    \item SunSpots: Wireless Sensor Network motes. See also 
    \href{http://en.wikipedia.org/wiki/Sun_SPOT}
        {this wikipedia article} - which includes some useful links. The sunspot
        includes the following:
        \subitem Microcontroller: 180 MHz 32 bit ARM920T core - 512K RAM - 4M
        Flash;
        \subitem 2.4 GHz IEEE 802.15.4 radio with integrated antenna;
        \subitem AT91 timer chip;
        \subitem USB interface;
        \subitem 2G/6G three-axis accelerometer;
        \subitem Temperature sensor;
        \subitem Light sensor;
        \subitem 8 tri-color LEDs;
        \subitem 6 analog inputs;
        \subitem 2 momentary switches;
        \subitem 5 general purpose I/O pins;
        \subitem 4 high current output pins;
        \subitem 3.7V rechargeable 750 mAh lithium-ion battery;
    \item jobot: 
    \item Hemisson: Predesigned robot with two wheels for mobility, that includes:
        \subitem a PIC16F877 microcontroller 20MHz CPU clock, 8bit, 8K words
        program memory;
        \subitem two DC motors for independent control of two wheel. Open loop
        control without encoders;
        \subitem eight IR ambient light sensors;
        \subitem six IR obstacle detection sensors;
        \subitem two line detection sensors;
        \subitem a standard 9V (PP3) battery connector;
        \subitem serial port with DB9 connector;
        \subitem a TV remote receiver;
        \subitem a buzzer;
        \subitem four LEDs;
        \subitem four programmable switches;
        \subitem an extension bus for extra modules;
    \item Linux, Mac OS X, Windows XP and Windows 7 driven PCs;
    \item USB to RS232 / DB9 converter. Make: K\"onig;
        

\end{itemize}

\section{Software}
% }}}

% }}}

\section{results} % {{{
% }}}

\section{assumptions} % {{{

% }}}

\newpage
\appendix

\section{Appendix A: microcontroller program source} % {{{
\begin{verbatim}
\end{verbatim}
% }}}

\section{Appendix B: sunspot source} % {{{

% }}}

\end{document}
